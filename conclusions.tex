\section{Supertree Estimation}

FastRFS and ASTRID are both effective methods for supertree estimation. FastRFS with SIESTA is more accurate than other supertree methods and can scale to datasets with thousands of taxa. ASTRID can run on even larger datasets, with tens of thousands of taxa, although it is slightly less accurate than FastRFS in some cases. The most promising avenues of future work have to do with improving FastRFS's search space. Currently, the most accurate version of FastRFS uses MRL as a subroutine to expand its search space, but this is computationally intensive on large datasets. Replacing this with ASTRID may allow for larger and faster analyses without sacrificing accuracy. FastRFS uses subroutines from ASTRAL to compute its search space, but these are designed for a species tree context. There may be ways to compute a search space that are better suited for the supertree context, perhaps using OCTAL \cite{christensen2018octal} to complete input trees and analyzing their clades.

\section{Species Tree Estimation}

ASTRID is among the most accurate methods for species tree estimation. It is also by far the fastest coalescent-aware summary method. Improvements to ASTRID are possible, both in terms of accuracy and in terms of speed. ASTRID's sample complexity is known to be theoretically limited by variance in its internode distance matrix \cite{roch2018variance}. Since variance estimates of the average distance matrix are easy to calculate, it may be possible to incorporate these into the distance-based estimation methods used in the second stage of ASTRID. Some methods, like BIONJ \cite{gascuel1997bionj}, can already take into account variance estimates and would be good candidates to test this approach. From a performance perspective, ASTRID should be fairly straightforward to parallelize, and this could reduce the amount of time used to estimate the average distance matrix. 

SVDquest presents an improvement over the implementation of SVDquartets in PAUP*. In most cases, however, either SVDquartets method is less accurate than CA-ML or summary methods like ASTRAL and ASTRID. Newer versions of PAUP* have added additional local search heuristics for improving its estimate; it may be possible to combine these with SVDquest's tree to find an even better tree. There may also be regions of parameter space, perhaps with even higher ILS or even shorter loci than those studied in Chapter \ref{chapter:svdquest}, where SVDquest does actually outperform its competitors. 

Further research into species tree estimation methods with both HGT and ILS will be important for larger analyses, simply due to the fact that datasets with more species are more likely to display both of these processes. It seems as though methods that are effective under ILS are also effective under ILS and HGT, but more datasets should be tested, including those with ``highways'' of HGT.

Finally, the most effective strategies for large scale estimation will likely involve combinations of methods. These could range from using methods like ASTRID or MRL to expand the search space of ASTRAL, FastRFS, or SVDquest in various ways, or using divide and conquer methods to allow SVDquest to run on larger datasets. 